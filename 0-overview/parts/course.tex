% course.tex

%%%%%%%%%%%%%%%%%%%%
\begin{frame}{}
  \begin{center}
    \red{\bf \large 什么是``计算机数学''?}

    \pause
    \vspace{1em}
    \blue{\bf \large ``Mathematics for Computer Science''} \\[5pt]
    \teal{(\textsf{\bf math4cs})}
    \fig{width = 0.40\textwidth}{figs/mcs-book}
  \end{center}
\end{frame}
%%%%%%%%%%%%%%%%%%%%

%%%%%%%%%%%%%%%%%%%%
\begin{frame}{}
  \begin{center}
    \red{\bf \large What does ``Computer Science'' study?}

    \pause
    \vspace{2em}
    \begin{quote}
      Computer science focuses on methods involved in
      \blue{design, specification, programming, verification,
      implementation and testing} of \teal{human-made computing systems}.
    \end{quote}

    \pause
    \vspace{2em}
    \blue{\large math4cs is \purple{\bf model-and-proof} oriented.}
  \end{center}
\end{frame}
%%%%%%%%%%%%%%%%%%%%

%%%%%%%%%%%%%%%%%%%%
\begin{frame}{}
  \begin{center}
    \red{\bf \large ``计算机数学''在哪里?}
    \fig{width = 0.60\textwidth}{figs/math-tree}

    \pause
    \blue{\bf \large 计算机数学是个大杂烩, 啥都学点儿, \xout{啥都没学好}}
  \end{center}
\end{frame}
%%%%%%%%%%%%%%%%%%%%

%%%%%%%%%%%%%%%%%%%%
\begin{frame}{}
  \begin{center}
    {\large 分班教学 \\ (共 xxx 班)}

    \vspace{0.60cm}
    {\large \red{授课内容与作业可能有出入}, 不影响考试与成绩分配}
  \end{center}
\end{frame}
%%%%%%%%%%%%%%%%%%%%

%%%%%%%%%%%%%%%%%%%%
\begin{frame}{}
  \begin{center}
    {\large 平时作业 {\it vs.} 期中测试 {\it vs.} 期末测试}

    \[
      3 \quad:\quad 3 \quad:\quad 4
    \]
    \[
      4 \quad:\quad 3 \quad:\quad 3
    \]

    \vspace{0.80cm}
    \blue{\large 弹性制}
  \end{center}
\end{frame}
%%%%%%%%%%%%%%%%%%%%

%%%%%%%%%%%%%%%%%%%%
\begin{frame}{}
  \begin{center}
    每周二、周四下午 14:00 发布作业 \qquad 下周四晚 22:00 前提交作业

    \vspace{1.00cm}
    每次作业按 \blue{10} 分计算

    \vspace{0.50cm}
    \red{\bf 迟交:} 周四\red{前}向助教登记, 可延长两天, 总分 \blue{8} 分

    \vspace{0.50cm}
    \cyan{({\bf 助教:} 、、)}
  \end{center}
\end{frame}
%%%%%%%%%%%%%%%%%%%%

%%%%%%%%%%%%%%%%%%%%
\begin{frame}{}
  \begin{center}
    \teal{``教学立方''课程邀请码: PLD8QKTZ}

    \begin{columns}
      \column{0.50\textwidth}
        \fig{width = 0.80\textwidth}{figs/qrcode-dm-class}
      \column{0.50\textwidth}
        \pause
        \fig{width = 0.30\textwidth}{figs/tex}
        \vspace{-0.30cm}
        \fig{width = 0.95\textwidth}{figs/math4cs-problem-sets-github}
    \end{columns}

    \vspace{0.50cm}
    \teal{\small \url{https://github.com/courses-at-hnu-by-hfwei/math4cs-problem-sets}}
  \end{center}
\end{frame}
%%%%%%%%%%%%%%%%%%%%

%%%%%%%%%%%%%%%%%%%%
\begin{frame}{}
  \fig{width = 0.50\textwidth}{figs/yuefa}
\end{frame}
%%%%%%%%%%%%%%%%%%%%

%%%%%%%%%%%%%%%%%%%%
\begin{frame}{}
  \begin{center}
    \red{\Large \bf 非必要, 不点名}
  \end{center}
\end{frame}
%%%%%%%%%%%%%%%%%%%%

%%%%%%%%%%%%%%%%%%%%
\begin{frame}{}
  \begin{center}
    \red{\bf \Large 非必要, 不迟到}

    \pause
    \vspace{0.60cm}
    \blue{\bf \large 尽量吃早餐, 但不可以在教室吃早餐}
  \end{center}
\end{frame}
%%%%%%%%%%%%%%%%%%%%

%%%%%%%%%%%%%%%%%%%%
\begin{frame}{}
  \begin{center}
    \red{\bf \Large \xout{非必要}, 不抄袭; 一经发现, 后果严重}

    \pause
    \vspace{1.50cm}
    \blue{\bf \Large 当次作业计0分; 总评扣10分}
  \end{center}
\end{frame}
%%%%%%%%%%%%%%%%%%%%

%%%%%%%%%%%%%%%%%%%%
\begin{frame}{}
  \begin{center}
    QQ 群号: \blue{\bf 108 745 6358}

    \vspace{1em}
    \fig{width = 0.40\textwidth}{figs/math4cs-qq-qrcode}
  \end{center}
\end{frame}
%%%%%%%%%%%%%%%%%%%%

%%%%%%%%%%%%%%%%%%%%
\begin{frame}{}
  \begin{center}
    \fig{width = 0.40\textwidth}{figs/dm-structures-ch}

    \vspace{0.30cm}
    {\large 教材不重要, 听讲更重要}
  \end{center}
\end{frame}
%%%%%%%%%%%%%%%%%%%%

%%%%%%%%%%%%%%%%%%%%
\begin{frame}{}
  \begin{center}
    \fig{width = 0.50\textwidth}{figs/mcs-book}

    \vspace{0.30cm}
    {\large 其它参考书随课程进度安排}
  \end{center}
\end{frame}
%%%%%%%%%%%%%%%%%%%%

%%%%%%%%%%%%%%%%%%%%
\begin{frame}{}
  \begin{center}
    \fig{width = 0.60\textwidth}{figs/github-lectures}
    \teal{\url{https://github.com/courses-at-nju-by-hfwei/discrete-math-lectures}}
  \end{center}
\end{frame}
%%%%%%%%%%%%%%%%%%%%