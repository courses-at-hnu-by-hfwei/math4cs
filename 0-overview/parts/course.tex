% course.tex

%%%%%%%%%%%%%%%%%%%%
\begin{frame}{}
  \begin{center}
    \red{\bf \large 什么是``计算机数学''?}

    \pause
    \vspace{1em}
    \blue{\bf \large ``Mathematics for Computer Science''} \\[5pt]
    \teal{(\textsf{\bf math4cs})}

    \pause
    \fig{width = 0.40\textwidth}{figs/mcs-book}
  \end{center}
\end{frame}
%%%%%%%%%%%%%%%%%%%%

%%%%%%%%%%%%%%%%%%%%
\begin{frame}{}
  \begin{center}
    \red{\bf \large What does ``Computer Science'' study?}

    \pause
    \vspace{2em}
    \begin{quote}
      Computer science focuses on methods involved in
      \blue{design, specification, programming, verification,
      implementation and testing} of \teal{human-made computing systems}.
    \end{quote}

    \pause
    \vspace{2em}
    \blue{\large math4cs is \purple{\bf model-and-proof} oriented.}
  \end{center}
\end{frame}
%%%%%%%%%%%%%%%%%%%%

%%%%%%%%%%%%%%%%%%%%
\begin{frame}{}
  \begin{center}
    \red{\bf \large ``计算机数学''在哪里?}
    \fig{width = 0.60\textwidth}{figs/math-tree}

    \pause
    \blue{\bf \large 计算机数学是个大杂烩, 啥都学点儿, \xout{啥都没学好}}
  \end{center}
\end{frame}
%%%%%%%%%%%%%%%%%%%%

%%%%%%%%%%%%%%%%%%%%
\begin{frame}{}
  \begin{center}
    {\large 分班教学 (共 \blue{\bf 9} 个班级)}

    \vspace{0.60cm}
    {\large \red{拔尖班: 独立授课, 独立考核}}

    \vspace{1em}
    \pause
    \fig{width = 0.30\textwidth}{figs/confident-student}
  \end{center}
\end{frame}
%%%%%%%%%%%%%%%%%%%%

%%%%%%%%%%%%%%%%%%%%
\begin{frame}{}
  \begin{center}
    {\large 平时作业 {\it vs.} 课堂测验 {\it vs.} 期中测试 {\it vs.} 期末测试}

    \[
      1 \quad:\quad 3 \quad:\quad 2 \quad:\quad 4
    \]
  \end{center}
\end{frame}
%%%%%%%%%%%%%%%%%%%%

%%%%%%%%%%%%%%%%%%%%
\begin{frame}{}
  \begin{center}
    \red{\bf 每周四}下午 14:00 发布平时作业 \qquad \red{下周四}晚 22:00 前提交作业

    \vspace{1.00cm}
    取最高的 \blue{\bf 12} 次作业成绩计入总评

    \vspace{0.50cm}
    \red{\bf 请按时提交, 过时不补}

    \vspace{0.50cm}
    \cyan{({\bf 助教:} 罗熙辰)}
  \end{center}
\end{frame}
%%%%%%%%%%%%%%%%%%%%

%%%%%%%%%%%%%%%%%%%%
\begin{frame}{}
  \begin{center}
    \teal{``教学立方''课程邀请码: PLD8QKTZ}

    \begin{columns}
      \column{0.50\textwidth}
        \fig{width = 0.80\textwidth}{figs/qrcode-dm-class}
      \column{0.50\textwidth}
        \pause
        \fig{width = 0.30\textwidth}{figs/tex}
        \vspace{-0.30cm}
        \fig{width = 0.95\textwidth}{figs/math4cs-problem-sets-github}
    \end{columns}

    \vspace{0.50cm}
    \teal{\small \url{https://github.com/courses-at-hnu-by-hfwei/math4cs-problem-sets}}
  \end{center}
\end{frame}
%%%%%%%%%%%%%%%%%%%%

%%%%%%%%%%%%%%%%%%%%
\begin{frame}{}
  \begin{center}
    约\red{\bf 每两周一次}课堂测验 (提前通知时间与测验范围)

    \vspace{1em}
    取最高的 \blue{\bf 6} 次课堂测验成绩计入总评

    \fig{width = 0.30\textwidth}{figs/academic-fortress}

    \red{\bf 请准时参加, 不安排补考}
  \end{center}
\end{frame}
%%%%%%%%%%%%%%%%%%%%

%%%%%%%%%%%%%%%%%%%%
\begin{frame}{}
  \fig{width = 0.50\textwidth}{figs/yuefa}
\end{frame}
%%%%%%%%%%%%%%%%%%%%

%%%%%%%%%%%%%%%%%%%%
\begin{frame}{}
  \begin{center}
    \red{\Large \bf 非必要, 不点名}
  \end{center}
\end{frame}
%%%%%%%%%%%%%%%%%%%%

%%%%%%%%%%%%%%%%%%%%
\begin{frame}{}
  \begin{center}
    \red{\bf \Large 非必要, 不迟到}

    \pause
    \vspace{0.60cm}
    \blue{\bf \large 尽量吃早餐, 但不可以在教室吃早餐}
  \end{center}
\end{frame}
%%%%%%%%%%%%%%%%%%%%

%%%%%%%%%%%%%%%%%%%%
\begin{frame}{}
  \begin{center}
    \red{\bf \Large \xout{非必要}, 不抄袭; 一经发现, 后果严重}

    \pause
    \vspace{1.50cm}
    \blue{\bf \Large \xout{当次作业计 0 分;} 平时作业成绩扣 2 分, 扣完为止}
  \end{center}
\end{frame}
%%%%%%%%%%%%%%%%%%%%

%%%%%%%%%%%%%%%%%%%%
\begin{frame}{}
  \begin{center}
    QQ 群号: \blue{\bf 108 745 6358}

    \vspace{1em}
    \fig{width = 0.40\textwidth}{figs/math4cs-qq-qrcode}
  \end{center}
\end{frame}
%%%%%%%%%%%%%%%%%%%%

%%%%%%%%%%%%%%%%%%%%
\begin{frame}{}
  \begin{center}
    \fig{width = 0.45\textwidth}{figs/discrete-math-book-yang}

    \vspace{1em}
    {\large 授课内容不局限于教材, 认真听讲很重要}
  \end{center}
\end{frame}
%%%%%%%%%%%%%%%%%%%%

%%%%%%%%%%%%%%%%%%%%
\begin{frame}{}
  \begin{columns}
    \column{0.50\textwidth}
      \fig{width = 0.80\textwidth}{figs/dm-structures-ch}
    \column{0.50\textwidth}
      \fig{width = 0.80\textwidth}{figs/dm-applications-book}
  \end{columns}

  \begin{center}
    {\large 内容与习题偏简单, 略显琐碎; 可用于课前预习及课后基础练习}
  \end{center}
\end{frame}
%%%%%%%%%%%%%%%%%%%%

%%%%%%%%%%%%%%%%%%%%
\begin{frame}{}
  \begin{center}
    \fig{width = 0.50\textwidth}{figs/mcs-book}

    \vspace{1em}
    {\large \blue{推荐阅读}; 其它参考书随课程进度安排}
  \end{center}
\end{frame}
%%%%%%%%%%%%%%%%%%%%

%%%%%%%%%%%%%%%%%%%%
\begin{frame}{}
  \begin{center}
    \fig{width = 0.80\textwidth}{figs/math4cs-lectures-github}

    \vspace{1em}
    \teal{\url{https://github.com/courses-at-hnu-by-hfwei/math4cs-lectures}}

    \vspace{2em}
    {0-overview.pdf \qquad 0-overview-\red{\bf handout}.pdf}
  \end{center}
\end{frame}
%%%%%%%%%%%%%%%%%%%%