% axiom.tex

%%%%%%%%%%%%%%%
\begin{frame}{}
  \begin{center}
    \blue{\large \bf Axiomatic Systems (公理系统)}

    \vspace{1em}
    Self-contained (自包含; 不假于外物); Self-consistent (自洽; 内部无矛盾)

    \fig{width = 0.90\textwidth}{figs/axiomatic-systems-tree-full}
  \end{center}
\end{frame}
%%%%%%%%%%%%%%%

% %%%%%%%%%%%%%%%%%%%%%%%%%%%%%%
% \begin{frame}{}
%   \begin{center}
%     Peano 公理体系刻画了\red{\bf 自然数的递归结构}
%   \end{center}

%   \begin{definition}[Peano Axioms]
%     \begin{enumerate}[(1)]
%       \setlength{\itemsep}{6pt}
%       \item 0 是自然数;
%       \item 如果 $n$ 是自然数, 则它的后继 ${\bf S}n$ 也是自然数;
%       \item 0 不是任何自然数的后继;
%       \item 两个自然数相等当且仅当它们的后继相等;
%       \item \red{\bf 数学归纳原理:} 如果
%         \begin{enumerate}[(i)]
%           \setlength{\itemsep}{8pt}
%           \item $P(0)$ 成立;
%           \item 对任意自然数 $n$, 如果 $P(n)$ 成立, 则 $P(n+1)$ 成立。
%         \end{enumerate}
%         那么, $P(n)$ 对所有自然数 $n$ 都成立。
%     \end{enumerate}
%   \end{definition}
% \end{frame}
% %%%%%%%%%%%%%%%%%%%%%%%%%%%%%%

%%%%%%%%%%%%%%%
\begin{frame}{}
  \begin{columns}
    \column{0.50\textwidth}
      \fig{width = 0.55\textwidth}{figs/elements-ch}
    \column{0.50\textwidth}
      \fig{width = 0.50\textwidth}{figs/hilbert-geometry}
  \end{columns}

  \vspace{0.30cm}
  \begin{description}[(平行公设)]
    \item[直线公设] 两点决定一条直线
    \item[\gray{延伸公设}] 直线段可以向两端无限延伸
    \item[圆公设] 以任一点为圆心、任意长为半径, 可作一圆
    \item[角公设] 所有直角皆相等
    \item[平行公设] \red{过直线外一点, 可作且只可作一直线与此直线平行}
  \end{description}
\end{frame}
%%%%%%%%%%%%%%%

% %%%%%%%%%%%%%%%
% \begin{frame}{}
%   \begin{columns}
%     \column{0.50\textwidth}
%       \fig{width = 0.55\textwidth}{figs/elements-ch}
%     \column{0.50\textwidth}
%       \fig{width = 0.50\textwidth}{figs/hilbert-geometry}
%   \end{columns}

%   \vspace{0.30cm}
%   \begin{enumerate}[(1)]
%     \item To draw a straight \blue{line} from any \blue{point} to any point.
%     \item To extend a finite straight line continuously in a straight line.
%     \item To describe a circle with any center and radius.
%     \item That all right angles are equal to one another.
%     \item \red{The parallel postulate}.
%   \end{enumerate}
% \end{frame}
% %%%%%%%%%%%%%%%

%%%%%%%%%%%%%%%%%%%%
\begin{frame}{}
  \begin{exampleblock}{Axiomatic System for a \red{Four-point Geometry}}
    \blue{\bf \textit{Undefined terms:}} point, line, is on

    \vspace{0.50cm}
    \purple{\bf \textit{Axioms:}}
    \begin{enumerate}[(1)]
      \item There are exactly four \teal{points}.
      \item It is impossible for three \teal{points} to \teal{be on} the same line.
      \item For every pair of distinct \teal{points} $x$ and $y$, \\
        there is a unique \teal{line} $l$ such that $x$ \teal{is on} $l$
        and $y$ \teal{is on} $l$.
      \item Given a \teal{line} $l$ and a \teal{point} $x$ that is not \teal{on} $l$,
        there is a unique \\ \teal{line} $m$ such that $x$ \teal{is on} $m$
        and no \teal{point} on $l$ is also \teal{on} $m$.
    \end{enumerate}
  \end{exampleblock}

  \pause
  \vspace{0.30cm}
  \begin{theorem}
    There are at least two distinct lines.
  \end{theorem}
\end{frame}
%%%%%%%%%%%%%%%

%%%%%%%%%%%%%%%
\begin{frame}{}
  \begin{center}
    \fig{width = 0.60\textwidth}{figs/syntax-semantics}

    \vspace{1em}
    Syntax {\it vs.} Semantics (语法与语义对立统一)
  \end{center}
\end{frame}
%%%%%%%%%%%%%%%

%%%%%%%%%%%%%%%%%%%%
\begin{frame}{}
  \begin{center}
    \fig{width = 0.70\textwidth}{figs/point-loop}

    \[
      \text{point}: \red{\cdot} \qquad
      \text{line}: \red{\bigcirc} \qquad
      \text{is on}: \red{\bigodot}
    \]
  \end{center}
\end{frame}
%%%%%%%%%%%%%%%

%%%%%%%%%%%%%%%
\begin{frame}{}
  \begin{center}
    \red{\bf \large 你还能想到哪些公理系统? 不一定是数学方面的!}
  \end{center}

  \pause
  \begin{columns}
    \column{0.40\textwidth}
      \fig{width = 0.95\textwidth}{figs/three-body-book}
      \begin{center}
        \red{\bf 宇宙社会学}
      \end{center}
    \column{0.60\textwidth}
      \fig{width = 0.95\textwidth}{figs/three-body-axioms}
  \end{columns}
\end{frame}
%%%%%%%%%%%%%%%

%%%%%%%%%%%%%%%
\begin{frame}{}
  \begin{center}
    本课程将介绍\red{\bf 三个公理系统}:

    \vspace{1em}
    \blue{逻辑、集合论、\gray{图论}、群论 (抽象代数)}
  \end{center}
\end{frame}
%%%%%%%%%%%%%%%

%%%%%%%%%%%%%%%%%%%%
\begin{frame}{}
  \begin{center}
    \red{\bf \large 什么样的推理是正确的?}

    \fig{width = 0.70\textwidth}{figs/euler-logic}
  \end{center}
\end{frame}
%%%%%%%%%%%%%%%%%%%%

%%%%%%%%%%%%%%%%%%%%
\begin{frame}
  \begin{center}
    \fig{width = 0.50\textwidth}{figs/see-you-next-time}
  \end{center}
\end{frame}
%%%%%%%%%%%%%%%%%%%%

%%%%%%%%%%%%%%%
% \begin{frame}{}
%   \begin{center}
%     {\Large Axiomatic Set Theory \blue{(ZFC)}}
%   \end{center}
%
%   \vspace{0.80cm}
%   \begin{columns}
%     \column{0.45\textwidth}
%       \fig{width = 0.50\textwidth}{figs/Zermelo}{\centerline{Ernst Zermelo (1871--1953)}}
%     \column{0.45\textwidth}
%       \fig{width = 0.48\textwidth}{figs/Fraenkel}{\centerline{Abraham Fraenkel (1891--1965)}}
%   \end{columns}
% \end{frame}
%%%%%%%%%%%%%%%

% %%%%%%%%%%%%%%%%%%%%
% \begin{frame}{}
%   \begin{exampleblock}{\red{四点几何}的公理系统}
%     \blue{\bf \textit{未定义术语:}} 点、线、点在线上

%     \vspace{0.50cm}
%     \purple{\bf \textit{公理:}}
%     \begin{enumerate}[(1)]
%       \item 有且仅有四个\teal{点}
%       \item 任意三\teal{点}不共\teal{线}。
%       \item 对于每一对不同的\teal{点} $x$ 和 $y$,
%         存在一条唯一的\teal{线} $l$,使得 $x$ \teal{在线} $l$ 上
%         且 $y$ \teal{在线} $l$ 上。
%       \item 给定一条\teal{线} $l$ 和一个不在\teal{线} $l$ 上的\teal{点} $x$,
%         存在一条唯一的\teal{线} $m$,使得 $x$ \teal{在线} $m$ 上
%         且\teal{线} $l$ 上的任何\teal{点}都不在\teal{线} $m$ 上。
%     \end{enumerate}
%   \end{exampleblock}

%   \pause
%   \vspace{0.30cm}
%   \begin{theorem}
%     至少存在两条不同的\teal{线}。
%   \end{theorem}
% \end{frame}
% %%%%%%%%%%%%%%%