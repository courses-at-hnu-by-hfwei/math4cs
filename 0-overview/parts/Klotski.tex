% Klotski.tex

% %%%%%%%%%%%%%%%%%%%%
% \begin{frame}
%   \begin{exampleblock}{Klotski Puzzle (华容道)}
%     \fig{width = 0.45\textwidth}{figs/Klotski}
%   \end{exampleblock}
% \end{frame}
% %%%%%%%%%%%%%%%%%%%%

% %%%%%%%%%%%%%%%%%%%%
% \begin{frame}
%   \begin{exampleblock}{Klotski Puzzle (华容道; 中国版本)}
%     \fig{width = 0.40\textwidth}{figs/HuaRongDao}
%   \end{exampleblock}
% \end{frame}
% %%%%%%%%%%%%%%%%%%%%

%%%%%%%%%%%%%%%%%%%%
\begin{frame}
  \begin{exampleblock}{\href{https://youtu.be/YI1WqYKHi78}{15 Puzzle (数字华容道)}}
    \begin{columns}
      \column{0.30\textwidth}
        \fig{width = 0.95\textwidth}{figs/15-puzzle-wiki}
      \column{0.30\textwidth}
        \fig{width = 0.95\textwidth}{figs/15-puzzle-unsolvable}
      \column{0.40\textwidth}
        \fig{width = 0.80\textwidth}{figs/15-puzzle-init}
    \end{columns}
  \end{exampleblock}

  \pause
  \vspace{0.50cm}
  \begin{center}
    \red{\large Is it solvable? \quad How to solve it? \quad How fast can we solve it?}
  \end{center}
\end{frame}
%%%%%%%%%%%%%%%%%%%%

%%%%%%%%%%%%%%%%%%%%
\begin{frame}{}
  \begin{center}
    It uses \blue{\bf permutation groups} in \blue{\bf group theory}.
  \end{center}

  \begin{columns}
    \column{0.50\textwidth}
      \fig{width = 0.80\textwidth}{figs/15-puzzle-paper}
    \column{0.50\textwidth}
      \pause
      \fig{width = 0.80\textwidth}{figs/stay-tuned}
  \end{columns}
\end{frame}
%%%%%%%%%%%%%%%%%%%%