% overview.tex

%%%%%%%%%%%%%%%%%%%%
\begin{frame}{}
  \fig{width = 0.80\textwidth}{figs/welcome-again}
\end{frame}
%%%%%%%%%%%%%%%%%%%%

%%%%%%%%%%%%%%%%%%%%
\begin{frame}{}
  \begin{center}
    {\bf \Large 计算机(离散)数学}

    \pause
    \vspace{0.50cm}
    研究\blue{\bf 离散对象的结构、性质、操作等}的数学分支 (\purple{\bf 大杂烩})
  \end{center}
\end{frame}
%%%%%%%%%%%%%%%%%%%%

%%%%%%%%%%%%%%%%%%%%
\begin{frame}{}
  \begin{center}
    {\bf \large \red{四大主题:} 逻辑、集合论、图论、抽象代数 (群论)}

    \fig{width = 0.50\textwidth}{figs/math-tree}

    {\bf \blue{支流遍布:} 组合与计数、数论、(离散)概率}
  \end{center}
\end{frame}
%%%%%%%%%%%%%%%%%%%%

%%%%%%%%%%%%%%%%%%%%
\begin{frame}{}
  \begin{center}
    {\large 关于计算机(离散)数学的古老传说:}

    \vspace{0.80cm}

    \red{\LARGE 我太难了}

    \vspace{0.60cm}

    \red{\LARGE 啥用没有}
  \end{center}
\end{frame}
%%%%%%%%%%%%%%%%%%%%

%%%%%%%%%%%%%%%%%%%%
\begin{frame}{}
  \begin{center}
    \red{\Large 真得有那么难吗?}

    \pause
    \vspace{0.80cm}
    {\large 确实蛮难的: 知识点多而分散、概念抽象}
  \end{center}
\end{frame}
%%%%%%%%%%%%%%%%%%%%

%%%%%%%%%%%%%%%%%%%%
\begin{frame}{}
  \begin{center}
    \red{\Large 真得没啥用吗?}

    \pause
    \vspace{0.80cm}
    {\large 太基础, 用了但不自觉 (\teal{逻辑})}

    \vspace{0.50cm}
    {\large 浅尝辄止, 想用但用不上 (\teal{群论})}
  \end{center}
\end{frame}
%%%%%%%%%%%%%%%%%%%%

%%%%%%%%%%%%%%%%%%%%
\begin{frame}{}
  \begin{center}
    \red{\Large 将计算机(离散)数学看作一门语言,一套工具}

    \vspace{1.00cm}
    {\large 培养形式化描述问题的能力}

    \vspace{0.60cm}
    {\large 培养做严格证明的能力}
  \end{center}
\end{frame}
%%%%%%%%%%%%%%%%%%%%

%%%%%%%%%%%%%%%%%%%%
\begin{frame}{}
  \begin{center}
    本课程最重要的主题: \red{\large \bf 证明}

    \vspace{1em}
    \fig{width = 0.35\textwidth}{figs/show-me-your-proof}
  \end{center}
\end{frame}
%%%%%%%%%%%%%%%%%%%%

%%%%%%%%%%%%%%%%%%%%
\begin{frame}{}
  \begin{columns}
    \column{0.24\textwidth}
      \fig{width = 0.95\textwidth}{figs/proof-trivial}
    \column{0.24\textwidth}
      \fig{width = 0.95\textwidth}{figs/proof-left}
    \column{0.24\textwidth}
      \fig{width = 0.95\textwidth}{figs/proof-trivial-left}
    \column{0.24\textwidth}
      \fig{width = 0.95\textwidth}{figs/proof-margin}
  \end{columns}
\end{frame}
%%%%%%%%%%%%%%%%%%%%

%%%%%%%%%%%%%%%%%%%%
\begin{frame}{}
  \begin{center}
    凡是``跳步''的证明, 一律默认是错的!!!

    \fig{width = 0.50\textwidth}{figs/proof-steps-colorful}
  \end{center}
\end{frame}
%%%%%%%%%%%%%%%%%%%%